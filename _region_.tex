\message{ !name(ECE742Project.tex)}\documentclass{article}
\usepackage{amsmath}

\title{ECE 742 Final Project}
\author{
  Michelle King
  \and
  Suraj Suri
  }

\begin{document}

\message{ !name(ECE742Project.tex) !offset(179) }


 \begin{multline*}
  D_{z}^{n+1}(i,j) = -\frac{\sigma_{x}\Deltat-2\epsilon_{0}\kappa_{x}}{\sigma_{x}\Delta t+2\epsilon_{0}\kappa_{x}}D_{z}^{n}(i,j) \\
 +\frac{2\epsilon_{0}\Delta t}{\Delta_{x}(\sigma_{x}\Delta t+2\epsilon_{0}\kappa_{x})}(H_{y}^{n+1/2}(i+1/2,j)-H_{y}^{n+1/2}(i-1/2,j)) \\
 -\frac{2\epsilon_{0}\Delta t}{\Delta_{y}(\sigma_{x}\Delta t+2\epsilon_{0}\kappa_{x})}(H_{x}^{n+1/2}(i,j+1/2)-H_{x}^{n+1/2}(i,j-1/2))
 \end{multline*}

 
Update Components for E\\
\\
Update Components for B\\
Update Components for H

\subsection{Graded Conductivity}

Reflection Factor
\[R(\theta)=\exp^{-2\eta cos(\theta)\int_{0}^{d}\sigma_{x}(x)dx\]

  Where $\sigma_{x}$ is the graded conductivity of the PML material.\\
  $\theta$ is the angle of incidence of the wave. So steeper angles of $\theta$
  will result in higher values of reflection error.\\
We want to minimize relfection R but also make sure the wave decays completely
in the PML boundary layer.

We are going to compare the error for different types of grading profiles.
And/or we can use different values in the grading profile 

\subsubsection{Polynomial grading}
Where the graded conductivity is:
\[\sigma_{x} = (\frac{x}{d})^{m} \sigma_{x,max}\]
And the graded value for $\kappa_{x}$ is:
\[\kappa_{x}=1+(\kappa_{x,max}-1)(\frac{x}{d})^{m}\]
Reflection factor simplifies to 

\subsubsection{Geometric grading}
Where the graded conductivity is:
\[\sigma_{x} =(g^{\frac{1}{\Delta}})^{x} \sigma_{x,0}\]

$\sigma_{x,0}$ is the conductivity at the surface of the PML.\\
g is a scaling factor. Nearly optimal: $2 \leq g \leq 3$\\
$\Delta$ is spacing of FDTD lattice.\\

And the graded value for $\kappa_{x}$ is:
\[\kappa_{x}=[(\kappa_{max})^{\frac{1}{d}}g^{\frac{1}{\Delta}}]^{x}\]

\section{Code}


\section{Error Analysis}
Insert Error Analysis Here

PMLs are exact for continuous functions, but error is introduced for discrete
functions. Having a large step discontinuity can

\subsubsection{Error of Polynomial Grading}

\subsubsection{Error of Geometric Grading}

\section{Fix me: Bibliography}
Susan's book - third edition

\end{document}
\message{ !name(ECE742Project.tex) !offset(-77) }
