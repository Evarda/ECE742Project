\documentclass{article}

\title{ECE 742 Final Project}
\author{
  Michelle King
  \and
  Suraj Suri
  }

\begin{document}
\maketitle
\section{Theory}
\subsection{PML}
Perfectly Matched Layer (PML) boundary conditions are absorbing boundary
conditions. PML BCs decay the wave within a boundary layer at the edge of the
simulation. The edge of the simulation BC can be implemented as PEC.
Well-implemented PML BCs completely decay the wave from the time it enters the
boundary layer to the time after it reflects and attempts to leave.

\subsection{Graded Conductivity}

Reflection Factor
\[R(\theta)=\exp^{-2\eta cos(\theta)\int_{0}^{d}\sigma_{x}(x)dx\]

Where $\sigma_{x}$ is the graded conductivity of the PML layer material.
We want to minimize relfection R but also make sure the wave decays completely
in the PML boundary layer.

We are going to compare the error for different types of grading profiles.

\subsubsection{Polynomial grading}
Where the graded conductivity is:
\[\sigma_{x} = (\frac{x}{d})^{m} \sigma_{x,max}\]
And the graded value for $\kappa_{x}$ is:
\[\kappa_{x}=1+(\kappa_{x,max}-1)(\frac{x}{d})^{m}\]

\subsubsection{Geometric grading}
Where the graded conductivity is:
\[\sigma_{x} =(g^{\frac{1}{\Delta}})^{x} \sigma_{x,0}\]
And the graded value for $\kappa_{x}$ is:
\[\kappa_{x}=[(\kappa_{max})^{\frac{1}{d}}g^{\frac{1}{\Delta}}]^{x}\]

\section{Code}


\section{Error Analysis}
Insert Error Analysis Here

PMLs are exact for continuous functions, but error is introduced for discrete functions. Having a large step discontinuity can 

\section{Fix me: Bibliography}
Susan's book - third edition

\end{document}